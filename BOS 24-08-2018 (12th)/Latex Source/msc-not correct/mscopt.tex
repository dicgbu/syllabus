\section
{\scshape Details of DSE-I/DSE-II/DSE-III: Discipline Specific Electives}

These courses if floats may be offered as open electives for other departments. 

%\begin{multicols}{\cols}
\credit{3}
\lecture{2}
\tutorial{0}
\lab{2}

\coursename{Machine Learning for Data Science}
\section{\dsccourseinfo}

Prerequisites: Basic linear algebra and calculus, introductory-level courses in probability and statistics. Comfort with a programming language (e.g., Matlab, Python) will be essential for completing the homework assignments.

This course provides an introduction to supervised and unsupervised techniques for machine learning. We will cover both probabilistic and non-probabilistic approaches to machine learning. Focus will be on classification and regression models, clustering methods, matrix factorization and sequential models. Methods covered in class include linear and logistic regression, support vector machines, boosting, K-means clustering, mixture models, expectation-maximization algorithm, hidden Markov models, among others. We will cover algorithmic techniques for optimization, such as gradient and coordinate descent methods, as the need arises.

References Books: 

[1] T. Hastie, R. Tibshirani and J. Friedman, The Elements of Statistical Learning, Second Edition, Springer.

[2] C. Bishop, Pattern Recognition and Machine Learning, Springer.

[3] H. Daume, A Course in Machine Learning, Draft. (http://ciml.info/)


%\coursename{Measure and Integration}
%\section{\dsccourseinfo}
%
%Review of Riemann Integral, Riemann-Stieltjes Integral. Lebesgue Measure; Lebesgue Outer Measure; Lebesgue Measurable Sets. Measure on an arbitrary sigma-Algebra; Measurable Functions; Integral of a Simple Measurable Function; Integral of Positive Measurable Functions. Lebesgue’s Monotone Convergence Theorem; Integrability; Dominated Convergence Theorem; Lp-Spaces. Differentiation and Fundamental theorem for Lebesgue integration. Product measure; Statement of Fubini’s theorem.  
%
% 
%
%References Books  
%
%[1] G. D. Barra, Measure and Integration, Wiley Eastern, 1981.  
%
%[2] Terence Tao, An Introduction to Measure Theory  
%
%[3] W. Rudin, Real and Complex Analysis, Third edition, McGraw-Hill, International Editions, 1987. 
%
%[4] I. K. Rana, An Introduction to Measure and Integration, Second Edition, Narosa, 2005.  
%
%[5] D. L. Cohn, Measure Theory , Birkhauser, 1997.  
%
%[6] H. L. Royden, Real Analysis, Third edition, Prentice-Hall of India, 1995. 


\coursename{Graph Theory}
\credit{3}
\lecture{3}
\tutorial{0}
\lab{0}
\section{\dsccourseinfo}
 

Graphs, Sub graphs, path and circuits, connected graphs, disconnected graphs and component, Euler graphs, various operation on graphs, Hamiltonian paths and circuits, the traveling sales man problem. Trees and fundamental circuits, fundamental circuits, algorithms of primes, Kruskal and Dijkstra Algorithms. Cuts sets and cut vertices, fundamental circuits and cut sets , connectivity and separability, network flows, planer graphs, Vector space of a graph and vectors, basis vector, cut set vector, circuit vector, circuit and cut set verses subspaces, orthogonal vectors and subspaces, incidence matrix of graph, sub matrices of A(G), circuit matrix, cut set matrix, path matrix and relationships among A, B, and C fundamental circuit matrix and rank of B, adjacency matrices, rank- nullity theorem . Coloring and covering and partitioning of a graph, Directed graphs, some type of directed graphs, Directed paths, and connectedness, Euler digraphs, trees with directed edges, fundamental circuits in digraph, matrices A, B and C of digraphs adjacency matrix of a digraph, counting of labeled and unlabeled trees, Polya's theorem. 

 

References Books  

[1]  N. Deo, Graph Theory, PHI. 

[2]  J. A. Bondy, U.S.R. Murthy, Graph theory with application, North Holland Publications, New York.  

[3]  R. Diestel,, Graph Theoty, Springer Verlag 


\coursename{Differential Geometry}
\section{\dsccourseinfo}
 
Geometry of curves, parameterization, arc length, curvature, torsion, serret frenet equation, global properties of curves in plane. Extrinsic geometry of surfaces, parameterization, tangent plane, differential, first and second fundamentals forms, curves in surface, normal and geodesic curvature of curves. Intrinsic geometry of surfaces, frames and frame fields, covariant derivatives and connection, Riemannian Metric, Gaussian curvature, Fundamental forms and the equations of gauss and coddazi-mainardi. Geometry of geodesics, Exponential map, geodesic polar coordinates, Fermi coordinates along a curve property of geodesic, Jacobi fields, and convex neighborhood. Global result from surfaces, The Gauss Bonnet Theorem, Hopf-Rinnoe theorem, cut points and conjugate points, the Bonnet-Myers theorem.   

 

References Books  

[1] M. D. Carmo, Differential Geometry of curves and surfaces, Prentice Hall. 

[2] B. O’Niell, Elementry Differential Geometry, Academic Press.  

[3] J. A. Thorpe, Elementry Topics in Differential Geometry, Springer. 

\coursename{Optimization Techniques}
\section{\dsccourseinfo}

Prerequisite: Operations Research

Nonlinear programming: Convex sets and Convex functions, definition and basic properties, Differentiable convex and concave functions, Minima and Maxima of Convex and Concave functions, The Fritz John and Karush-Kuhn-Tucker Optimality Conditions, Quadratic programming with linear constraints, Wolfe’s Algorithm, Complementary Pivot Algorithm, Separable Programming, Geometric Programming, Search Techniques: Line search for unimodal functions, Fibonacci method of search, Golden Section Search, Steepest descent method, conjugate direction, conjugate direction method, Powell’s Quadratic Interpolation Method.
 

References Books  

[1] J.C. Pant, Introductio to optimization: Operations Research,Jain Brothers, New Delhi, 2002. 

[2] D. G. Luenberger, Yinyu Ye, Linear and nonlinear programming, Springer, 2008. 

[3] M. S. Bazaraa, H. D. Sherali, C. M. Shetty, Nonlinear Programming:Theory and Algorithms, John Wiley and Sons,2006. 

[4] A. Ravindran, D. Phillips, and J. Solberg, Operations Research: Principles and Practice, 2nd Ed., Wiley India, 2007. 

 

\coursename{Elliptic Curves}
\section{\dsccourseinfo}

Rational points on Plane Curves, The group law on a cubic curve, Regular functions; the Riemann-Roch theorem, Basic Theory of Elliptic Curves, The Weierstrass equation for an elliptic curve, The Group Law, Projective Space and the Point at Infinity, Elliptic Curves in Characteristic 2, Singular curves,Reduction of an elliptic curve modulo p, Algorithms for elliptic curves, Torison points, Elliptic Curves over Finite Fields 

The Discrete Logarithm Problem, Elliptic Curve Cryptography. 

 

 \textbf{Reference Books} 

[1] J.S. Milne, Elliptic Curves, BookSurge Publishers, 2006 

[2] J. H. Silverman, Arithmatics of Elliptic Curves, Springer, 2006 

[3] Lawrence C. Washington, Elliptic Curves, Taylor \& Francis Group, 2008 

 
\coursename{Discrete Mathematics}
\section{\dsccourseinfo}

 Propositional logic, Proof techniques: forward proof, proof by contradiction, contrapositive proofs, proof of necessity and sufficiency. principles of mathematical induction, Strong Induction. Introduction to counting: Basic counting techniques, Pigeonhole principle, Application of pigeon-hole principle, Advanced Counting Technique- inclusion and exclusion, Application of inclusion and exclusion, Introduction to recurrence relation and generating function. Relations and their properties, equivalence relations, closures, Transitive Closure and Warshall’s Algorithms, functions, order relation and structure, partially ordered sets, Lattices, Introduction to Graphs: Shortest Path Problem, Dijkstra’s Shortest path algorithm, Trees, Minimum Spanning Tree, Application of Trees, Tree Traversal, Finite Boolean Algebra, functions on Boolean Algebra.   

 

References Books  

[1] K. H. Rosen, Discrete Mathematics and its Applications, Tata McGraw-Hill.  

[2] C. L. Liu, Elements of Discrete Mathematics, Tata McGraw-Hill.  

[3] T. Koshy, Discrete Mathematics with Applications, Elsevier.  

[4] R. P. Grimaldi, Discrete and Combinatorial Mathematics, Pearson Education, Asia.  

[5] B. Kolman, R. Busby and S. C. Ross, N. Rehman, Discrete Mathematical Structures, Pearson Prentice Hall. 

 

  
\coursename{Fractional Calculus}
\section{\dsccourseinfo}


Gamma function, Mittag-Leffler function, Weight function, Grunwald Letnikov. Grnwald Letnikov Fractional Derivatives. Riemann-Liouville Fractional Derivatives. Some Other Approaches. Geometric and Physical Interpretation of Fractional Integration and Fractional Differentiation. Sequential Fractiona Derivatives. Left and Right Fractional Derivatives. Properties of Fractional Derivatives. Laplace Transforms of Fractional Derivatives. Fourier Transforms of Fractional Derivatives. Mellin Transforms of Fractional Derivatives. Fractional Differential Equation of a General Form. Existence and Uniqueness Theorem as a Method of Solution. Dependence of a Solution on Initial Conditions. The Laplace Transform Method. Standard Fractional Differential Equations. Sequential Fractional Differential Equations. The Mellin Transform Method. Power Series Method. Babenko’s Symbolic Calculus Method. Method of Orthogonal Polynomials. Numerical Evaluation of Fractional Derivatives. Approximation of Fractional Derivatives. The ”Short-Memory” Principle.

References Books  

[1] I. Podlubny, I. Petras, B. M. Vinagre, P. O’Leary, L. Dorcak, Analogue realizations of fractional-order controllers. Nonlinear Dynamics, vol. 29, no. 14, 2002, pp. 281–296.  

[2] I. Podlubny, Geometric and physical interpretation of fractional integration and fractional differentiation. Fractional Calculus and Applied Analysis, vol. 5, no. 4, 2002, pp. 367– 386.  

[3] I. Podlubny, Matrix approach to discrete fractional calculus. Fractional Calculus and Applied Analysis, vol. 3, no. 4, 2000, pp. 359–386.l 

   
\coursename{Applied Linear Algebra}
\section{\dsccourseinfo}

Review of Linear Algebra; LU-decomposition, QR-decomposition, The geometry of linear equations, least squares approximation, Eigen values and Eigen vectors of $e^{At}$, Gershgorin circle theorem and its applications. Courant-Fischer minimax and related Theorems. Markov matrices, defective matrices, Jordan form, Generalized eigenvectors, Singular value decomposition, sparse matrices and Iterative methods.  

Applications of Linear Algebra: Linear Programming, Fourier Series, Computer graphics, Difference equations and powers $A^k$; Differential equations and $e^{At}$. 

 

References Books  

[1] S. Axler, Linear Algebra done right, Springer 

[2] G. Strang, Introduction to Linear Algebra, 4th edition, Cambridge University Press India Pvt Ltd, 2009 

[3] G.  Strang, Linear Algebra and Its Applications, 4th edition, Brooks/Cole (Cengage Learning), 2006 

 
\coursename{Commutative Algebra}
\section{\dsccourseinfo}
 

Ring and Ideal: Zero-divisors, Nilpotent Elements, Prime, Maximal Ideals, Nilradical, Jacobson radical, Extension and Contraction of ideals, Nakayama Lemma. Localization: Ring and Module of fractions, Spec of a ring. Integral Dependence: integral dependence, going-up and going-down theorems, Valuation Rings. Chain Conditions: Noetherian and Artin rings. Dimension Theory: Graded ring, Hilbert function and Samuel function, Dimension of Noetherian ring.   

 

References Books  

[1] M. F. Atiyah and I. G. Macdonald, Introduction to Commutative Algebra, Addison-Wesley, 1969  

[2] H. Matsumura, Commutative Ring theory, Cambridge University Press, 1989.  

[3] D. S. Dummit \& R. M. Foote, Abstract Algebra, 2nd edition, John Wiley, 2008.  

[4] S. Lang, Algebra, revised 3rd edition, Springer (2004). 

 

\coursename{Fuzzy theory and applications}
\section{\dsccourseinfo}

 

Fuzzy Sets and Uncertainty: Uncertainty and information, fuzzy sets and membership functions, chance verses fuzziness, properties of fuzzy sets, fuzzy set operations, Fuzzy Relations: Cardinality, operations, properties, fuzzy cartesian product and composition, fuzzy tolerance and equivalence relations, forms of composition operation. Fuzzification and Defuzzification: Various forms of membership functions, fuzzification, defuzzification to crisp sets and scalars. Fuzzy Logic and Fuzzy Systems: Classic and fuzzy logic, fuzzy rule-based systems, graphical technique of inference, Development of membership functions: Membership value assignments: intuition, inference, rank ordering, neural networks, genetic algorithms, inductive reasoning. Fuzzy Arithmetic and Extension Principle: Functions of fuzzy sets, extension principle, fuzzy mapping, interval analysis, vertex method and DSW algorithm. Fuzzy Optimization: One dimensional fuzzy optimization, fuzzy concept variables and casual relations, fuzzy cognitive maps, agent-based models.   

 

References Books  

[1] Ross, T. J., “Fuzzy Logic with Engineering Applications”, Wiley India Pvt. Ltd., 3rd Ed. 2011  

[2] Zimmerman, H. J., “Fuzzy Set theory and its application”, Springer India Pvt. Ltd., 4th Ed. 2006  

[3] Klir, G. and Yuan, B., “Fuzzy Set and Fuzzy Logic: Theory and Applications”, Prentice Hall of India Pvt. Ltd. 2002  

\coursename{Design Optimization}
\section{\dsccourseinfo} 

%MA 033 (Design Optimization)                                                      Credits (L-T-P): 3(3-0-0)  



Formulation of a design optimization problems, Local and Global Minima, Pareto optimum, optimality Conditions, Iterative function minimization methods, Iterative methods for constrained problems, Exploratory Methods, Multi-criteria optimization methods, Decision making problem, Introduction to Genetic Algorithm, Parameters of Genetic algorithms, chromosome representation, Encoding,  Selection mechanisms, Evolutionary operators, Fitness function, Handling constraints, Penalty function strategy, tournament selection in constrained optimization, Methods of selecting a set of Pareto optimal solutions, Distance method, Pareto set distribution method, Constraint tournament selection method for multi-criteria optimization, Application of Genetic algorithm to solve design optimization problem. 



 \textbf{Reference Books} 

[1] Andrzej Osyczka, “Evolutionary Algorithms for Single and Multicriterian Design Optimization” Volume 79 of Studies in Fuzziness and Soft Computing, Physica-Verlag Heidelberg,  25-Sep-2001 - Business Economics - 218 pages. 

[2] Kalyanmoy Deb, “Optimization for Engineering Design: Algorithms and Examples” Prentice Hall India Learning Private Limited; second edition edition (1995). 

[3] S. S. Rao, “Engineering Optimization: Theory and Practice”, New Age International Publishers; Third edition (1 January 2013) 

\coursename{Computational Mathematics with Python}
\lecture{2}
\lab{2}
\section{\dsccourseinfo}

Elementary programing concepts: Arithmetic expressions, for-loops, logical expressions, if statements, functions and classes. These concepts are taught exclusively using mathematical/technical problems and examples. Mathematical Manipulations: Setting up matrices, solving linear problems, solving differential equations, finding roots, eigenvalues, resonances, without going into the mathematical details. More advanced concepts such as generators are presented and a basic introduction to the ideas of object-oriented programming will be given.

References Books 

[1] Dive into Python, Free available on Python website.

%\coursename{Numerical Solutions of ODE}
%\lecture{2}
%\lab{2}
%\section{\dsccourseinfo}
%
%
%Numerical methods for initial value problems: Convergence analysis of the general explicit one-step method, derivation of classical Runge-Kutta methods, Runge-Kutta methods of order greater than four, implicit Runge-Kutta methods. Predictor-corrector methods:Milne’s methods, Adams-Bashforth Method, absolute stability and accuracy of predictor-corrector methods. Boundary value problems, BVP as an eigenvalue problem. Linear multi-step methods: Construction of linear multistep methods, Convergence, Order and error constant, Local and global truncation error, Consistency and zero-stability, maximum order of zero-stable methods.  
%
% 
%
%References Books  
%
% 
%
%[1] J.D. Lambert, Computational Methods in Ordinary Differential Equations. Wiley, Chichester, 1991. 
%
%[2] R.J. Leveque, Finite Difference Methods for Ordinary and Partial Differential Equations. SIAM,     Philadelphia, 2007.  
%
%[3] G. Sewell, Numerical solution of ordinary and partial differential equations, 2ed., Wiley, 2005.  
%
%[3] R. S. Gupta, Elements of Numerical Analysis, Macmillan, 
%
% 
%\coursename{Numerical Solutions of PDE}
%\section{\dsccourseinfo}
%%MA 012 ()         Credits (L-T-P): 3(2- 0- 2)  
%
% 
%
%Finite Differences for Linear Equations: Introduction to finite difference formulae, finite difference approximations to derivatives, Linear Parabolic equations-explicit and implicit schemes, Crank Nicholson Method, local truncation error, compatibility, consistency and convergence of the difference methods, Stability analysis (Energy method, Matrix method and Von-Neumann method); Gerschgorin‘s theorems, Lax equivalence theorem. Introduction to multi dimensional problems, Linear Hyperbolic equations, Finite differences, Theoretical concepts of Stability and consistency, order of accuracy, upwind, Lax Friedrichs and Lax-Wendroff schemes. Linear Elliptic equations-Finite difference schemes, Alternating direction methods.  
%
% 
%
%Finite Difference schemes for Nonlinear Equations: One dimensional scalar conservation laws, Review of basic theory, Solutions of the Riemann problem and entropy conditions. First order schemes like Lax Friedrichs, Godunov, Enquist Osher and Roe's scheme. Convergence results, entropy consistency and Numerical viscosity. Introduction to higher order schemes-Lax Wendroff scheme, Upwind schemes of Van Leer, ENO schemes, Central schemes, Relaxation methods. Convection-Reaction-Diffusion equations, Extension to the above methods. Splitting schemes for multi dimensional problems. Introduction to finite elements methods, 
%
%  
%
% 
%
%References Books  
%
%[1] G. D. Smith, Numerical solutions to Partial Differential Equations, Brunel University, Clarendon Press, Oxford, 1985. 
%
%[2] J. Strikwerda, Finite Difference Scheme and Partial Differential Equations, SIAM, 1989. 
%
%[3] G. Sewell, Numerical solution of ordinary and partial differential equations, 2ed., Wiley, 2005.  
%
%[3] R. S. Gupta, Elements of Numerical Analysis, Macmillan, 

\coursename{Optimal Control to PDEs}
\section{\dsccourseinfo}

Introduction, optimal control, convex problems, nonconvex problems, Basic concepts for the finite-dimensional case, Linear-quadratic elliptic control problems, Weak solutions to elliptic equations, Linear mappings, Existence of optimal controls, Differentiability in Banach spaces, Adjoint operators, First-order necessary optimality conditions, Construction of test examples,The formal Lagrange method, The adjoint state as a Lagrange multiplier’ Higher regularity for elliptic problems, Regularity of optimal controls, Linear-quadratic parabolic control problems, Parabolic optimal control problems, Necessary optimality conditions, Optimal control of semilinear elliptic equations, A semilinear elliptic model problem, Existence of optimal controls,control-to-state operator, Necessary optimality conditions, Pontryagin’s maximum principle, Second-order optimality condition. 
 

References Books  

[1] Fredi Tröltzsch, Optimal Control of Partial Differential Equations Theory, Methods and Applications, American Mathematical Society Providence, Rhode Island, 2005. 

   [2]  J.C.Lions, Optimal Control of Systems Governed by Partial Differential Equations, Springer. 

 
\coursename{Advanced Optimization Techniques}
\lecture{3}
\lab{0}
\section{\dsccourseinfo}
%MA-014 (Advanced Optimization Techniques)           Credits (L-T-P): 3(3- 0- 0)  

 
Generalized convex functions, Quasi convex functions, Quasi concave functions, Pseudo convex functions, Pseudo concave functions, Feasible direction, Cone of feasible direction, Necessary and Sufficient conditions, Fritz John Optimality conditions, Kuhn- Tucker optimality conditions, weak duality theorem, Saddle point optimality criteria, Generalized Gorden theorem, Lemkis Complementary Pivot algorithm, Linear Fractional Problem, Charnes-Cooper method, method of Gilmore and Gomory, Fletcher Reeves method, Davidon-Fletcher Powell Method, Constrained Optimization, Sequential unconstrained minimization technique, penalty and barrier function methods.   

 

References Books  

[1]  D. G. Luenberger, Yinyu Ye, Linear and nonlinear programming,Springer, 2008. 

[2]  O. L. Mangasarian, Nonlinear Programming, McGraw Hill, 1969. 

[3] M. S. Bazaraa, H. D. Sherali, C. M. Shetty, Nonlinear Programming: Theory and Algorithms, John Wiley and Sons, 2006 

 
\coursename{Applied Approximation}
\section{\dsccourseinfo}
%MA-015 ()            Credits (L-T-P): 3(3- 0- 0) 

Weierstrass Approximation theorem, Least square approximation, Minimax approximation, orthogonal polynomials, approximation with rational functions, Pade’s approximation, Use of Pade’s approximation. Various results related to existence of Adomian polynomials and its implementation in Matlab. Fundamental Adomian Decomposition method, various modified version of Adomian decomposition method, Applications of MADM to IVP, BVP. 

 

References Books  

[1] G. Adomian, Nonlinear Linear stochastic operators equations (1986). Academic Press, Inc (London)  

[2] A. M Wazwaz, Partial Differential equation and Solitary waves theory (2009).Higher Education Press and Springer. 

 
\coursename{Perturbation Methods}
\section{\dsccourseinfo}
 %MA-016 (Perturbation Methods)          Credits (L-T-P): 3 (3- 0- 0) 

 

 Asymptotic expansion and approximations, Asymptotic solution of algebraic and transcendental equations, Introduction to the asymptotic solution of differential equations. Singular and Regular Perturbations, Perturbed second order differential equations, Dimensional analysis, Initial and boundary value problems, Partial differential equations, Error estimation. Multiple scales, Overview of multiple scales and averaging, The first order twoscale approximation, Higher order approximations. Methods of WKB type, Introductory examples, The formal WKB expansion without turning points, Ray methods 

 

References Books  

[1] J. A. Murdock, Perturbations Theory and Methods, John Wiley and Sons, 1991.  

[2] M. H. Holmes, Introduction to Perturbation Methods, Springer Verlag, 1995.  

[3] A. H. Nayfeh, Perturbation Methods, John Wiley and Sons, 2000. 

 
\coursename{Dynamic Meteorology}
\section{\dsccourseinfo}
%MA-017 (Dynamic Meteorology)          Credits (L-T-P): 3(3- 0- 0) 

 

Basic Concepts, hydrostatic equilibrium, hydrostatic stability and convection, mean annual heat balance, fundamental forces, equations of motion in rotating and non rotating coordinate frames, scale analysis, basic conservation laws, spherical coordinates, geostrophic approximation, hydrostatic balance, static stability, circulation and vorticity conservation of potential vorticity, Rossby adjustment theory, quasi-geostrophic equations, omega equations, hydrodynamic instability.  

 

References Books  

[1]  S. L. Hess, Introduction to Theoretical Meteorology,, Krieger Pub. Co. Press. 

[2] James, C. Holton, An Introduction to Dynamic Meteorology,, Academic Press, 3rd edition. 

 

 
\coursename{Coding Theory}
\section{\dsccourseinfo}
 %MA 018 (Coding Theory)            Credits (L-T-P): 3(3- 0- 0)  

 

The communication channel, the coding problem, types of codes,block codes,error detecting and error -correcting codes, linear codes, the Hamming metric,descritption of linear block codes by matrices , dual codes, standard array, syndrome,step-by step, decoding, modular representation , error - correction capatiltes of linear codes,Hamming sphere packing bound, Hamming codes, Golay Codes, Perfect Codes,  

Basics and Algebraic Codes. Linear Block Codes: Generator and parity-check matrices, Minimum Distance, Sydrome decoding, Bounds on minimum distance. Cyclic Codes: Finite fields, Binary BCH codes, RS codes. 

 

 \textbf{Reference Books} 

[1]Ramond Hill, A first Course in Coding Theory , oXxford University press, 1990 

[2]w.w. Peterson and E.J.Weldon Jr. Error- Correcting Codes, Mit Prees, Cambridge, Masssachusetts,1972 

 

 \coursename{Analytic Number Theory}
\section{\dsccourseinfo}

%MA 019 (Analytic Number Theory)          Credits (L-T-P): 3(3- 0- 0) 

  

Arithmetical Functions: Mobius, Euler totient functions and their relation, Mangoldt function, Dirichlet product, Dirichlet and Mobius inverse formulae, Multiplicative function, Inverse of Completely multiplicative function. Congruences: Residue classes and Complete residue classes, Euler-Fermat Theorem, Polynomial Congruences, Chinese remainder theorem and its applications. Quadratic residue and Quadratic Reciprocity law.  

 

References Books  

[1] T. M. Apostol, Introduction to Analytic number Theory, Springer, 1980. 

[2] J. P. Serre, A course in Arithmetic, Springer, 1973. 

[3] J. Stillwell, Elements of Number Theory, Springer, 2003 

 
\coursename{Symmetries}
\section{\dsccourseinfo}
%MA 20 (Symmetries)                       Credits (L-T-P): 3(3- 0- 0)   

 

Symmetry of plane figures, Group of motions of plane, Finite group of motions, Discrete group of motions, Abstract symmetries, Operation of cosets, Counting formula, Permutation representations, Finite subgroups of rotation group. Bilinear Form: Definition, Symmetric forms, Orthogonality, Geometry associated to positive form, Hermitian forms, Conic and Quadrics, Skew-symmetric forms. 

 

References Books  

[1] M. Artin, Algebra, Prentice Hall of India (1991).  

[2] D. S. Dummit, R. M. Foote, Abstract Algebra, 2nd edition, John Wiley. 2008  

 

\coursename{Numerical Analysis with Programming}
\lecture{2}
\lab{2}
\section{\dsccourseinfo} 

%MA 021  (Numerical Analysis with Programming)          Credits (L-T-P): 3(2- 0- 2) 

 

Errors in computation: Floating point representation, Rounding, Chopping, Arithmetic Operation,  Truncation and Taylor Series, Linear Equations and Eigenvalue Problem:  Ill-conditioned systems and conditioned number;  iterative methods, Eigen value and Eigen vectors; Given’s method and Householder method for symmetric matrices; Polynomial methods, Power method; Non linear equations: Method of Successive Approximation, Aitken’s method, Newton-Raphson, convergence analysis. Interpolation:  Lagrange Interpolation, Truncation error in polynomial interpolation, Inverse interpolation,  Hermite's interpolation, Piecewise polynomial interpolation: Cubic spline interpolation, Numerical differentiation, Numerical integration:   Newton-Cotes Quadrature Formulae, Romberg integration, Gaussian Quadrature (Gauss-Legendre Quadrature Formula. 

Laboratory: Implementation of methods discussed in above syllabus. 

References Books 

[1] R. S. Gupta, Elements of Numerical Analysis Macmilan, 2009. 

[2] Srimanta Pal, Numerical Methods: Principles, Analysis and Algorithms (Oxford Higher Education) 

[3] K. E. Atkinson, Introduction to Numerical Analysis, 2ndEdn., John Wiley, 1989. 

[4] Curtis F. Gerald, Patrick O. Wheatley, Applied NumericalAnalysis, Pearson, 2003. 

[5] R.L. Burden and J.D. Faires, Numerical Analysis, 7th Edit.,Thomson, New York, 2007 

 

 \coursename{Dynamic Oceanography}
 \lecture{3}
\lab{0}
\section{\dsccourseinfo}
%MA 022(Dynamic Oceanography)              Credits (L-T-P): 3(3- 0- 0) 

  

Basic Hydrodynamic equations of motion and continuity, physical and chemical properties of ocean water, composition of sea water, salinity, density, thermal expansion of sea water, mass transport and free surface equation; The equation of motion in Oceanography, Reynolds Stresses, stability and double diffusion, steady motion in sea, Unsteady motions and their solutions, Currents without friction: Geostrophic flow, Currents with friction; wind driven circulation, Ekman Solution to the equation of motion with friction, Sverdrups solution for the wind driven solution, General Approach to numerical modeling of ocean circulation. 

 

References Books 

[1] G. L. Pickard and W.J. Emery, Descriptive Physical Oceanography,: Pergamon Press.  

[2] J. Pedlosky, Geophysical Fluid Dynamics,. Springer Verlag. 

[3] Stephen Pond, and G. L. Pickard, Introductory Dynamical Oceanography,. Pergamon Press. 

 
\coursename{Foundations of Cryptography}
\section{\dsccourseinfo}
%MA  023 (Foundations of Cryptography)           Credits (L-T-P): 3(3- 0- 0) 

 

Introduction, Perfect secrecy, One-time-pad encryption, Indistinguishability-based secrecy, Concrete security and asymptotic security, Computational indistinguishability and computationally secure symmetric-encryption. Pseudo-random number generators (PRNG, or stream ciphers). The computational one-time pad. The cascading construction. Forward security for PRNGs. Pseudo-random functions (PRFs) and pseudo-random permutations (PRPs and strong PRPs/blockciphers). The Feistel transform 

and the design of DES. Applications of PRFs/s-PRPs: Chosen-Plaintext Attacks (CPA) security. Modes of operation. Data origin and Message Authentication Codes (MACs). Data integrity and cryptographic hash functions (collision resistant vs. universal hash functions). Asymmetric cryptography: The key exchange problem. Merkle puzzles. The Diffie-Hellman Key Exchange protocol. Easy and hard problems in 
Z∗P
. Quadratic residuosity in 
Z∗P
. The PohligHellman cipher and Shamir’s no-key protocol. Public-key encryption: Security notions and applications. ElGamal encryption. Easy and hard problems in 
Z∗n
. Quadratic residuosity in 
Z∗n
. 

Rabin encryption. The RSA family of permutations. Chosenciphertext(CCA) security. RSA-OAEP encryption. Hybrid encryption. Digital signatures and the notion of Public-Key Infrastructure. 

 

References Books 

 [1] J. Katz and Y.a Lindell, Introduction to Cryptography, Springer, 2010 

[2] O. Goldreich, Foundations of Cryptography, CRC Press (Low Priced Edition Available), Part 1 and Part 2. 

[3] H. Delfs, H.t Knebl, Introduction to Cryptography, Principles and Applications, Springer Verlag. 

[4] W. Mao, Modern Cryptography, Theory and Practice, Pearson Education (Low Priced Edition) 

[5] S. Goldwasser and M. Bellare, Lecture Notes on Cryptography, Available in http://citeseerx.ist.psu.edu 

 
\coursename{Quantum Computing}
\section{\dsccourseinfo}
%MA 024  (Quantum Computing)          Credits (L-T-P): 3(3- 0- 0) 

 

Mathematical foundations; quantum mechanical principles; quantum entanglement; reversible computation, qubits, quantum gates and registers; universal gates for quantum computing; quantum parallelism and simple quantum algorithms; quantum Fourier transforms and its applications, quantum search algorithms; elements of quantum automata and quantum complexity theory; introduction to quantum error correcting codes; entanglement assisted communication; elements of quantum information theory and quantum cryptography. 

 

References Books 

 [1] M. A. Nielsen and I. L. Chuang, Quantum Computation and Quantum Information, Cambridge University Press. 

[2] J. Gruska, Quantum Computing, McGraw-Hill. 

[3] Lecture notes by John Preskill and N. D. Mermin available in the Internet. 

 
\coursename{Computational Number Theory}
\section{\dsccourseinfo}
%MA 025 (Computational Number Theory)                          Credits (L-T-P): 3(3- 0- 0) 

 

Algorithms for integer arithmetic: Divisibility, gcd, modular arithmetic, modular exponentiation, Montgomery arithmetic, congruence, Chinese remainder theorem, Hensel lifting, orders and primitive roots, quadratic residues, integer and modular square roots, prime number theorem, continued fractions and rational approximations. Representation of finite fields: Prime and extension fields, representation 

of extension fields, polynomial basis, primitive elements, normal basis, optimal normal basis, irreducible polynomials. Algorithms for polynomials: Root-finding and factorization, Lenstra-Lenstra-Lovasz algorithm, polynomials over finite fields. Elliptic curves: The elliptic curve group, elliptic curves over finite fields, Schoof’s point counting algorithm. Primality testing algorithms: Fermat test, Miller-Rabin test, Solovay-Strassen test, AKS test. Integer factoring algorithms: Trial division, Pollard rho method, p-1 method, CFRAC method, quadratic sieve method, elliptic curve method. Computing discrete logarithms over finite fields: Baby-stepgiant-step method, Pollard rho method, Pohlig-Hellman method, index calculus methods, linear sieve method, Coppersmith’s algorithm 

 

References Books 

[1] V. Shoup, A Computational Introduction to Number Theory and Algebra, Cambridge University Press. 

[2] A. J. Menezes, editor, Applications of Finite Fields, Kluwer Academic Publishers. 

[3] J. H. Silverman and John Tate, Rational Points on Elliptic Curves, Springer International Edition. 

[4] D. R. Hankerson, A. J. Menezes and S. A. Vanstone, Guide to Elliptic Curve Cryptography, Springer-Verlag. 

[5] A. Das and C. E. Veni Madhavan, Public-key Cryptography: Theory and practice, Pearson Education Asia. 

[6] H. Cohen, A Course in Computational Algebraic Number Theory, Springer-Verlag. 

 
\coursename{Queuing Theory \& Stochastic Process}
\section{\dsccourseinfo}
%MA 026(Queuing Theory& Stochastic Process)              Credits (L-T-P): 3(3- 0- 0)  

 

Queuing Theory, Kendall notation, steady and transient state, Single server with finite and infinite capacity, multiple server with finite and infinite capacity, Random Process, Mean variance and correlation, Stationay random process, auto correlation, cross correlation, Binomial, Normal and poission process, birth and death process. 

  

References Books  

[1] Sunderapandian, V., Probability, Statistics and Queuing theory, PHI. 

 [2] Alexander M. Mood, Franklin A Graybill, Duane C. Boes, Introduction to the theory of Statistics, TMH. 

 [3] R. Nelson, Probability, Stochastic Process and Queuing theory, Springer, New York 

                  
\coursename{Sampling Theory \& Statistical Inference}
\section{\dsccourseinfo}
%MA 027 (Sampling Theory & Statistical Inference)        Credits (L-T-P): 3(3- 0- 0) 

 

Sampling theory, distribution of sample, statistic and sample moments, sample mean, law of large number, central limit theorem, sampling from normal distribution, parametric point estimation, method of moments, method of maximum likelihood, parametric interval estimation, parametric and nonparametric test. 

 

References Books  

 [1] Gupta and Kapoor, Fundamental of mathematical statistics, S. Chand. 

 [2] Alexander M. Mood, Franklin A Graybill, Duane C. Boes, Introduction to the theory of Statistics, TMH. 

 [3] R. Nelson, Probability, Stochastic Process and Queuing theory, Springer, New York 

 \coursename{Mathematical modeling with MATLAB}
 \lecture{2}
 \lab{2}
\section{\dsccourseinfo}
%MA 028 (Mathematical modeling with MATLAB)        Credits (L-T-P): 3(2- 0- 2) 

 

MATLAB Environment, M-Files, Scripts, MATLAB Functions, Program design and development, Mathematical Modeling, Proportionality and geometric similarity, Modeling with proportionality, Empirical modeling, Discrete system modeling, Time series model, Continuous system modeling, Differential equation solution, Regression models, Interpolation and series, Symbolic processing with MATAB: Symbolic expressions and algebra, Algebraic and transcendental equations, Linear programming: formulation and Solution. 

 

References Books  

[1] R. Illner et al. “Mathematical Modelling: A Case Studies Approach”, AMS, 2005.  

[2] E. Bender. “Introduction to Mathematical Modelling”,Dover, 2000.  

[3] Amos Gilat, “MATLAB, An Introduction with Applications”, 2009. 

[4] M.C. Ferris, O.L. Managasarian, Stephen J. Wright, “Linear Programming with MATLAB”, SIAM, 2007. 

 
%\coursename{Evolutionary Algorithms}
% \lecture{3}
% \lab{0}
%\section{\dsccourseinfo}
%%MA 029 (Evolutionary Algorithms)                                               Credits (L-T-P): 3(3- 0- 0)  
%
%Genetic Algorithms: Historical development, GA concepts – encoding, fitness function, population size, selection, crossover and mutation operators, along with the methodologies of applying these operators. Binary GA and their operators, Real Coded GA and their operators.  Particle Swarm Optimization: PSO Model, global best, Local best, velocity update equations, position update equations, velocity clamping, inertia weight, constriction coefficients, synchronous and asynchronous updates, Binary PSO. Memetic Algorithms: Concepts of memes, Incorporating local search as memes, single and multi memes, hybridization with GA and PSO, Generation Gaps, Performance metrics. Differential Evolution: DE as modified GA, generation of population, operators and their implementation. Artificial Bee Colony: Historical development, types of bees and their role in the optimization process. Multi-Objective Optimization: Linear and nonlinear multi-objective problems, convex and non – convex problems, dominance – concepts and properties, Pareto – optimality, Use of Evolutionary Computations to solve multi objective optimization, bi level optimization, Theoretical Foundations.  
%
% 
%
% 
%
%References Books  
%
% [1] Coello, C. A., Van Veldhuizen, D.A. and Lamont, G.B.: “Evolutionary Algorithms for solving Multi Objective Problems”, Kluwer.2002  
%
%[2] Deb, K.: “Multi-Objective Optimization using Evolutionary Algorithms”, John Wiley and Sons. 2002  
%
%[3]Deb, K.: “Optimization for Engineering Design Algorithms and Examples”, Prentice Hall of India. 1998 
%
%[4] Gen, M. and Cheng, R.: “Genetic Algorithms and Engineering Design”, Wiley, New York. 1997 
%
%   [5] Hart, W.E., Krasnogor, N. and Smith, J.E. : “Recent Advances in Memetic Algorithms”, Springer Berlin   Heidelberg, New York. 2005  
%
%[6] Michalewicz, Z.: “Genetic Algorithms+Data Structures=Evolution Programs”, Springer-Verlag, 3rd edition, London, UK. 

 
\coursename{Methods of Applied Mathematics}
 \lecture{3}
 \lab{0}
\section{\dsccourseinfo}
%MA 029

Pade's Approximation, Z-Transform, Mellin Transform, Laplace transform, definition, properties and evaluation of transforms, Convolution theorem for Z-transforms, Applications to integral equations. Laplace transform method for Partial differential equation. Adomian Decomposition method, and Modified Adomian decomposition method and their Comparison between alternative (Direct Computation method, successive approximation method). Integral-Differential Equations. Singular Integral Equations, Generalized Abel Integral Equation, Weakly-singular Volterra Equations.

Reference Books

[1] G. F. Roach, Greens Functions, Cambridge University Press, 1995.

[2] G. Adomian, Nonlinear Stochastic Operator Equations, Academic Press, INC

[3] G. A. Bliss, Calculus of Variations, Open Court Publishing, 1944

[4] O. Bolza, Lectures on the Calculus of Variations, Dover Publication, New York, 1961.

[5] J. A. Cochran, The Analysis of Linear Integral Equations, McGraw-Hill, 1972



 \coursename{Finite Element Methods for Partial Differential Equations}
\section{\dsccourseinfo}

%MA 030 ()      Credits (L-T-P): 3(3- 0- 0) 

 

Basic concepts of finite element methods; Elements of function spaces, Lax-Milgram theorem, piecewise polynomial approximation in function spaces, Construction of finite element spaces, Polynomial approximations and interpolation errors, Convergence analysis: Galerkin orthogonality and Ceas lemma, Bramble-Hilbert lemma, Aubin-Nitsche duality argument; Applications to elliptic, parabolic and hyperbolic equations, a priori error estimates, variational crimes, A posteriori error analysis reliability, efficiency and adaptivity, Computational experiments using CAS. 

 

 \textbf{Reference Books} 

[1] P. G. Ciarlet, The Finite Element Method for Elliptic Problems,North-Holland, 1978. 

[2] J. N. Reddy, An Introduction to Finite Element Method, Mc-Graw Hill, 1993. 

[3] S. C. Brenner, L. R. Scott, The Mathematical Theory of Finite Element Methods, 2nd edition, Springer, 2002. 

[4] Z. Chen, Finite Element Methods and Their Applications,Springer, 2005. 

[5] D. L. Logan, A First Course in the Finite Element Method,4th edition, Cenegage Learning, 2007. 

 
\coursename{Fluid Mechanics}
\section{\dsccourseinfo} 
%MA 031 ()       Credits (L-T-P): 3 (3- 0- 0) 

 

Introduction, Kinematics of Fluid flow, Laws of fluid motion, Inviscid incompressible flows two and three-dimensional motions, Lagrangian and Eulerian descriptions, inviscid compressible flows, Viscous incompressible flows, Reynolds transport theorem, Navier-Stokes equations of motion and some exact solutions, Flows at small Reynolds numbers, Boundary layer theory. 

\textbf{Reference Books} 

[1] A. J. Chorin, J. E. Marsden, A Mathematical Introduction to Fluid mechanics, Springer-Verlag, 1999. 

[2] P. K. Kundu, I. M. Cohen, Fluid Mechanics (3rd edition) Elsevier Science and Technology, 2002. 

[3] H. Schlichting, K. Gersten, Boundary Layer Theory Springer-Verlag, 1985. 

[4] F. Chorlton, Textbook of Fluid Dynamics 1st Edition, CBS Publisher, 2004 

[5] G. K. Batchelor, Introduction to Fluid dynamics, Cambridge University Press. 

 
\coursename{Computational Fluid Dynamics}
\section{\dsccourseinfo} 
%MA 032 ()                     Credits (L-T-P): 3 (3- 0- 0) 

Review of the governing equations of Incompressible viscous flows, Stream function, vorticity approach, artificial vorticity, transport equations, upwind differencing schemes, Primitive variables, Staggered grid, stability analysis, Dimensional analysis, pressure correction and vortex methods; Compressible inviscid flows, central schemes with combined and independent space time discretisation, Compressible viscous flows, Explicit, implicit and FTCS, BTCS methods, Grid generation: Structured and unstructured grid generation methods, Finite volume method: Finite volume method to convection-diffusion equations. 

\textbf{Reference Books} 

[1] C. A. J. Fletcher, Computational Techniques for Fluid Dynamics,Volume 1 \& 2, Springer Verlag, 1992. 

[2] C. Y. Chow, Introduction to Computational Fluid Dynamics,John Wiley, 1979. 

[3] M. Holt, Numerical Methods in Fluid Mechanics, SpringerVerlag, 1977. 

[4] H. J. Wirz, J. J. Smolderen, Numerical Methods in FluidDynamics, Hemisphere, 1978. 

[5] J. D. Anderson, Computational Fluid Dynamics, The Basics with Applications, McGraw-Hill, 1995 

