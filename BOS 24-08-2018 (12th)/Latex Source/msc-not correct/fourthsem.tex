\coursecode{MA 502}
\coursename{Mathematical Statistics}
\lecture{4}
\tutorial{0}
\lab{2}
\section{\courseinfo}

Prerequisites: Probability and Calculus

Probability: Quick review, Random Variables, Distribution functions, and Expectation, Probability Inequalities (Markov, Chebyshev, and Jensen), Moments and moment generating functions, Characteristic functions. Special Parametric Families of Univariate Distributions: Discrete Uniform, Binomial, Poisson distributions, Normal distribution, Exponential and Gamma distributions. Joint, marginal and conditional distributions, Covariance and correlation coefficient, stochastic Independence, Joint moment generating Function and moments. Methods of finding Distributions of functions of random variables: The Expectation Technique, The Cumulative distribution function technique, The Transformation technique. Sampling and sampling distributions: Sampling, Distribution of sample, Statistic, Sample moments, Sample Mean, Weak Law of large numbers, Strong Law of large numbers, Central limit theorem. Standard Errors: Standard error of moments, Standard error of Sample Mean. Parametric point estimation: Methods of finding estimators, Method of moments, Maximum likelihood method, Other methods, Properties of point estimators, unbiasedness, sufficiency, efficiency and consistency. Test of Hypotheses: Parametric and non-parametric Hypotheses, Simple and composite Hypotheses, Critical reasons and alternative Hypotheses, Power of a test, Most powerful test, Uniformly most powerful test, Tests on Mean and variance for a sampling from normal distribution (Z-Test, t-Test), Chi-square test of goodness of fit, Chi-square test of independence in contingency tables.

Supplementary Topics: 
The moment generating function technique for determination of distribution of functions of random variables.

Text Books: 

A. M. Mood, F. A. Graybill and D. C. Boes - Introduction to the Theory of Statistic, Third Edition, TMH, 2001.

References Books 

[1] Alan Stuart and Keith Ord – Kendall’s Advanced Theory of Statistics, Vol-1, Distribution Theory , Sixth Edition, John Wiley \& Sons, Reprint 2015.

[2] Alan Stuart and Keith Ord and Steven Arnold - Kendall’s Advanced Theory of Statistics, Vol – 2A, Classical Inference \& Linear Model, Sixth Edition, John Wiley \& Sons, Reprint 2015.

[3] W.W. Hines, D. C. Montgomery, D. M. Goldsman, C. M. Borror- Probability and Statistics in Engineering, Fourth Edition, Wiley India Pvt. Ltd., Reprint 2009.


\coursecode{MA 504}
\coursename{Evolutionary Algorithms}
\section{\courseinfo}

Genetic Algorithms: Historical development, GA concepts – encoding, fitness function, population size, selection, crossover and mutation operators, along with the methodologies of applying these operators. Binary GA and their operators, Real Coded GA and their operators.  Particle Swarm Optimization: PSO Model, global best, Local best, velocity update equations, position update equations, velocity clamping, inertia weight, constriction coefficients, synchronous and asynchronous updates, Binary PSO. Memetic Algorithms: Concepts of memes, Incorporating local search as memes, single and multi-memes, hybridization with GA and PSO, Generation Gaps, Performance metrics. Differential Evolution: DE as modified GA, generation of population, operators and their implementation. Artificial Bee Colony: Historical development, types of bees and their role in the optimization process. 

Supplementary Topics: Multi-Objective Optimization: Linear and nonlinear multi-objective problems, convex and non – convex problems, dominance – concepts and properties, Pareto – optimality, Use of Evolutionary Computations to solve multi objective optimization, bi level optimization, Theoretical Foundations. 

References Books 

[1] Coello, C. A., Van Veldhuizen, D.A. and Lamont, G.B.: “Evolutionary Algorithms for solving Multi Objective Problems”, Kluwer.2002 

[2] Deb, K.: “Multi-Objective Optimization using Evolutionary Algorithms”, John Wiley and Sons. 2002 

[3]Deb, K.: “Optimization for Engineering Design Algorithms and Examples”, Prentice Hall of India. 1998

[4] Gen, M. and Cheng, R.: “Genetic Algorithms and Engineering Design”, Wiley, New York. 1997

   [5] Hart, W.E., Krasnogor, N. and Smith, J.E. : “Recent Advances in Memetic Algorithms”, Springer Berlin   Heidelberg, New York. 2005 

[6] Michalewicz, Z.: “Genetic Algorithms+Data Structures=Evolution Programs”, Springer-Verlag, 3rd edition, London, UK.


\coursecode{MA 506}
\coursename{Numerical Solutions of ODE and PDE}
\section{\courseinfo}

Numerical methods for initial value problems: Convergence analysis of the general explicit one-step method, derivation of classical Runge-Kutta methods, Runge-Kutta methods of order greater than four, implicit Runge-Kutta methods. Predictor-corrector methods: Milne’s methods, Adams-Bashforth Method, absolute stability and accuracy of predictor-corrector methods. Boundary value problems, BVP as an eigenvalue problem. Linear multi-step methods: Construction of linear multistep methods, Convergence, Order and error constant, Local and global truncation error, Consistency and zero-stability, maximum order of zero-stable methods. 

Finite Differences for Linear Equations: Introduction to finite difference formulae, finite difference approximations to derivatives, Linear Parabolic equations-explicit and implicit schemes, Crank Nicholson Method, local truncation error, compatibility, consistency and convergence of the difference methods, Stability analysis (Energy method, Matrix method and Von-Neumann method); Gerschgorin‘s theorems, Lax equivalence theorem. Introduction to multi dimensional problems, Linear Hyperbolic equations, Finite differences, Theoretical concepts of Stability and consistency, order of accuracy, upwind, Lax Friedrichs and Lax-Wendroff schemes. Linear Elliptic equations-Finite difference schemes, Alternating direction methods. 

Finite Difference schemes for Nonlinear Equations: One dimensional scalar conservation laws, Review of basic theory, Solutions of the Riemann problem and entropy conditions. First order schemes like Lax Friedrichs, Godunov, Enquist Osher and Roe's scheme. Convergence results, entropy consistency and Numerical viscosity. Introduction to higher order schemes-Lax Wendroff scheme, Upwind schemes of Van Leer, ENO schemes, Central schemes, Relaxation methods. Convection-Reaction-Diffusion equations, Extension to the above methods. Splitting schemes for multi dimensional problems. Introduction to finite elements methods,
 

References Books 

[1] G. D. Smith, Numerical solutions to Partial Differential Equations, Brunel University, Clarendon Press, Oxford, 1985.

[2] J. Strikwerda, Finite Difference Scheme and Partial Differential Equations, SIAM, 1989.

[3] G. Sewell, Numerical solution of ordinary and partial differential equations, 2ed., Wiley, 2005. 

[4] R. S. Gupta, Elements of Numerical Analysis, Macmillan,

[5] J.D. Lambert, Computational Methods in Ordinary Differential Equations. Wiley, Chichester, 1991.

[6] R.J. Leveque, Finite Difference Methods for Ordinary and Partial Differential Equations. SIAM,     Philadelphia, 2007. 


\coursecode{MA 508}
\coursename{Information Theory and Coding}
\section{\courseinfo}

Prerequisite: Probability theory.

Entropy and Shannon’s First Theorem, Mutual information, Data compression, Huffman coding, Asymptotic equipartition property, Universal source coding, Information Channel and Channel Capacity, Differential entropy, Block codes and Convolutional codes, Error correcting codes, Linear Codes, Optimal Codes. block coding, convolutional coding, and Viterbi decoding algorithm
 
References Books 
 
 [1] G. A. Jones and J. M. Jones, “Information and Coding Theory,” Springer, ISBN 1-85233-622-6, 3rd Edition
 
 [2] T. M. Cover, J. A, Thomas, “Elements of information theory,” Wiely Interscience, 2 nd Edition, 2006/ 

[3] R. W. Hamming, “Coding and information theory,” Prentice Hall Inc., 1980.



\coursecode{MA 520}
\coursename{Project}
\lecture{0}
\tutorial{0}
\lab{2}

\section{\courseinfo}

A student will choose a topic on advanced mathematics related to but beyond first year courses. He may choose a research paper. The student will learn the chosen material under the supervision of a teacher. He will meet the supervisor regularly (at least once per week) to present the material (s)he has learned and will update about his/her progress. Two seminars with satisfactory comments on his/her topic are mandatory. Finally, the student is expected to write an expository report of about 10-20 pages on the topic of project and will present this work to a panel of examiners at the end of term.   
