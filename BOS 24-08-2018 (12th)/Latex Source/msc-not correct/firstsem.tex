\coursecode{MA 401}
\coursename{Linear Algebra}
\section{\courseinfo}

Prerequisite: Algebra on Matrices, Elementary Matrices and their properties, Application to System of linear equations, Rank of a matrix.

Vector Spaces over field F, Subspaces, Bases and Dimension, Some infinite-dimensional vector spaces, Linear maps and their matrix representation, Algebra of linear maps, Isomorphism, Lineal functional, Dual spaces, Double dual, Annihilator, Transpose of a linear map, Inner product spaces, Cauchy-Schwarz inequality, Orthogonality, Orthonormal bases, Gram-Schmidt Orthonormal process, Complex inner product spaces, Polynomial of matrices, Eigen-value and Eigen-vectors, Diagonalization of matrices, Invariant subspaces, Direct sum decomposition, Invariant direct sums, The primary decomposition theorem, Jordan canonical form, cyclic subspaces, Rational canonical form, Quotient spaces, Bilinear form, Positive definite, Symmetric form, Hermitian form.
Supplementary Topics: Modules, Tensor product, Rigid motions.

Text Book: 

K. Hoffman and R. Kunze, Linear Algebra. Prentice Hall of India.

Reference Books

[1] S. H. Friedberg, Arnold J Insel and Lawrence Spence, Linear Algebra, 4th ed., Pearson Education

[2] S. Lipschutz and M. Lipson, Linear Algebra, Schum’s series, McGraw Hill Education. 2005

[3] M. Artin, Algebra, PHI Learning Private Limited 2005.

\coursecode{MA 403}
\coursename{Abstract Algebra}
\section{\courseinfo}

Prerequisite: Basic group theory. Examples of Dihedral, symmetric, permutation, Quaternion and matrix groups:  $GL_n$, $SL_n$ etc. 

Isomorphism theorems, Cayley Theorem, Class equation, Sylow theorems. Rings Theory: Definitions with some standard examples, Basic properties, Types of rings, Matrix rings, Ideals, Prime and Maximal ideals, Quotient rings, Homomorphisms of rings, Field of fractions, Polynomial ring with some properties, Polynomial functions, symmetric polynomials, Gaussian rings, Principal ideal domains, Euclidean domains, Unique factorization domains, Gauss’ lemma, Irreducibility criterion. Field Extension: Basic theory of field extensions, Algebraic extensions, Splitting filed of a polynomial, Algebraic closure, Separable and inseparable extensions, Group of automorphisms, Galois group (Introduction). 

Supplementary Topics: Classification of groups of small orders up to 15. Classification of finitely generated abelian groups. 

Text Book

N. Jacobson, Basic Algebra-I, 2nd Edition, Dover Publication 2009.

References Books

[1] D. S. Dummit, R. M. Foote, Abstract Algebra, 3rd edition, Wiley India. 2014.

[2] J. B. Fralieigh, A first course in abstract algebra, Pearson, 1994.

[3] Michael Artin, Algebra, Prentice Hall of India (1991).

[4] Serge Lang, Algebra, revised 3rd edition, Springer (2004).

[5] J. A. Gallian, Contemporary Abstract Algebra, Narosa.


\coursecode{MA 405}
\coursename{Real Analysis}
\section{\courseinfo}

Prerequisite: Basic real analysis and linear Algebra.  

Construction of Real Number System; Dedekind's cut; Topology on R, Weierstrass Theorem, Heine-Borel Theorem, Connectedness, Definition and existence of Riemann Stieltjes integral, Properties  of the integral, Integration and Differentiation, Rectifiable curves, Sequence and series of functions: Point-wise and uniform convergence, Cauchy criterion for uniform convergence,  Uniform convergence and continuity, Uniform convergence and integration, Uniform convergence and differentiation, Equicontinuous families of functions, Arzela-Ascoli theorem, Stone-Weierstrass theorem, Special functions: Power series, Fourier series, Gamma functions, Calculus of several variables: Contraction principle, Inverse function theorem, Implicit function theorem, rank theorem, differentiation of integrals.

Text Book

W. Rudin, Principles of Mathematical Analysis, McGrawHill

References Books

[1] T. Apostol, Mathematical Analysis, Addison-Wesley.

[2] H. L. Royden, Real Analysis, Macmillan Publishing Compa

[3] S. Lang - Analysis.


\coursecode{MA 407}
\coursename{Ordinary Differential Equations}
\section{\courseinfo}


Prerequisite: Basic knowledge of calculus, linear Algebra and complex number system.  

Review of solution methods for first order as well as second order equations, Euler-Cauchy Equations, Variation of parameter method, Wronskian, fundamental solutions, matrix exponential solution, Qualitative properties of the solutions of second order ODE, Normal form, Strum comparison, Separation theorem sand oscillations, Initial Value Problems, Existence and uniqueness of solutions to first order equations: Picard’s Theorem, Lipschitz condition, Gronwell’s inequality, Power Series Method, Regular Singular Points, Frobenius Method, Boundary Value Problems, Orthonormal Functions, Sturm Liouville’s Problems, Regular Sturm Liouville’s Problems, Eigenvalues and Eigen Functions, Eigen Function Expansion, Singular Sturm Liouville’s Problems, Adjoint equations, Lagrange’s identity, Nonhomogeneous Boundary Value Problems, Green’s functions.

Supplementary Topics: System of ODE-Linear homogeneous system, Fundamental matrix, exponential of the matrix, Phase space, Nonlinear system, introduction of system analysis. 

Text Book

W. E. Boyce and R. C. Di Prima, Elementary Differential Equations and Boundary Value Problems, Wiley, 2000.

References Books

[1] G. F. Simmons, Differential equations with applications and Historical Notes, Second Edition, Mc-Graw Hill, 1991

[2] Martin Braun, Differential Equations and Their Applications: An Introduction to Applied Mathematics (Texts in Applied Mathematics, Vol. 11) (Springer)

[3] W. E. Boyce and R. C. Di Prima, Elementary Differential Equations and Boundary Value Problems, Wiley, 2000.

[4] E. A. Coddington, An Introduction to Ordinary Differential Equations, Dover Publications, 1989.

[5] S. L. Ross, Introduction to Ordinary Differential Equations,4th Edition, Wiley, 1989.

[6] G. Birkhoff and G. C. Rota, Ordinary Differential Equations, John Wiley \& Sons, 1989.

\coursecode{MA 409}
\coursename{Number Theory and Cryptography}
\lecture{3}
\tutorial{0}
\lab{4}
\section{\courseinfo}

Prerequisite: Basic knowledge number theory.  

Notion of Complexity Theory, Euclidean algorithm, The fundamental theorem of arithmetic, Factorization methods, Linear Diophantine equations. Congruences linear congruences, Chinese remainder theorem, Wilson’s, Fermat’s and Euler’s theorem, Euler’s Phi-function. Applications to Congruences (time permitting) divisibility tests. Classical Cryptosystems, Crypt analysis, Perfect Secrecy, Stream Ciphers, Block Ciphers, Hash Functions, Public-key cryptography: RSA, Implementation of RSA, Primality Testing, Factoring Algorithm. Discrete logarithmic Algorithms. Diffie Hellman Problem. Finite Field.

Supplementary Topics: DES, Zero knowledge protocol, Threshold cryptosystems, Practical aspects of Cryptography
	
Text Book

N. Koblitz, A Course in Number Theory and Cryptography, 2nd edition, Springer, 1994

References Books 

[1] D. Welsh, Codes and Cryptography, Oxford, 2000.

[2] J. Buchmann, Introduction to Cryptography, Springer

[3] Joseph H. Silverman, A Friendly Introduction to Number Theory, Pearson, 2013
