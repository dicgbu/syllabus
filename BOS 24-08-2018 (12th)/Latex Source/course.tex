%&pdflatex

\documentclass{article}
\usepackage{gbucourse,amsmath,amssymb}
\usepackage{soul}
\usepackage{multicol,graphicx}
\usepackage{fullpage}
\renewcommand{\refname}{\normalsize Books}
\newcommand{\cols}{1}
\usepackage{fmtcount}
%\pagewise
%\pagestyle{sideheading}
%\usepackage{compactbib}
%\usepackage{natbib}
%\setlength{\bibsep}{0.0pt}



% ADD THE FOLLOWING COUPLE LINES INTO YOUR PREAMBLE
\let\OLDthebibliography\thebibliography
\renewcommand\thebibliography[1]{
  \OLDthebibliography{#1}
  \setlength{\itemsep}{0pt plus 0.3ex}
}
\setlength{\parskip}{0.4em}
%%%%%%%%%%%%%%%%%%%55
\usepackage{color,colortbl}
\definecolor{Gray}{gray}{0.85}
%\includeonly{ma111}

\begin{document}

\begin{titlepage}

\begin{center}
{\Huge\bf Course Structures and Detailed Syllabuses }



\vspace{3cm}
{\Huge\bf M.Sc. and Ph.D. (Applied Mathematics)}\\


\vspace{0.5cm}
{\Large Applicable since session 2018-2019}

\vspace{3cm}
{\huge  12th Board of Studies }\\
\vspace{0.5cm}
{\Large (held on 24-08-2018)}

\vfill
\includegraphics[width=1.5in]{gbu_logo}

\vspace{1cm}
{\bf \LARGE {Department of Applied Mathematics}}\\
{\Large {School of Vocational Studies  and Applied Sciences}}\\
{\Large GAUTAM BUDDHA UNIVERSITY}\\
{\Large Greater Noida, UP-201312}

\vspace{1cm}
{\huge 2018-2019}
\end{center}
\end{titlepage}
\title{Mathematics Courses}
%\title{Gautam Buddha University \\ School of Vocational Studies and Applied Sciences\\ Mathematics Courses}
%\author{Department of Applied Mathematics \\ \textbf{B.Tech.+M.Tech./MBA} \\{ (Engineering/ICT/Biotechnology/Food Processing and Technology)}}
\date{WEF 2018-2019}
\maketitle
\vspace{-2cm}
\tableofcontents
\newpage
%\part{\normalsize B.Sc. \\(Applied Mathematics)}
%\add{bsc/ma111}
\section{Preamble}
The Program M.Sc. in Applied Mathematics was started in 2012 since the inception of the Department of Applied Mathematics. The progression of students belonging to this course is excellent. 

In the development of this course, we considered various stack holders e.g. students, faculty, industry feedback etc. To make course more applied we have added new modules of laboratory work. The laboratory works certainly improve the skills of our students. This program has duration of 2 years and in this duration, students will earn 90 credits to get his M.Sc. degree.

\section*{Minimum Eligibility criteria for admission:}

Candidates should have either one of the following degrees:

1.	Bachelor of Science in Mathematics under the 10+2+3/4 system with at least 55\% marks or equivalent

2.	B.Tech. or B.E. (in any engineering branch) with at least 6 out of 10 CGPA or equivalent.

For SC/ST and PWD candidates the qualifying degree is relaxed to 50\% (or 5.5 CGPA out of 10)

\section*{Selection Procedure}

Candidates will be selected on the basis of GBU admission policy.

{\vfill}

\section{Members of the BOS}
\begin{enumerate}
	\item Prof Niranjan Melkania, (Chairperson) Dean SOVSAS, GBU
	\item Prof Riddhi Shah, (External Expert) JNU
	\item Dr. Amit K. Awasthi, Member (Head Applied Mathematics), GBU
	\item Dr. Fahad Zulfeqarr, Member (Applied Mathematics), GBU
\end{enumerate}


\newpage
\part{\normalsize M.Sc. (Applied Mathematics)}
\add{msc/structure}
\newpage
%%%%%%%%%%%%%%%%%%%%%%%%%%%%%%%%%%%%%%%%%%%%%%%%%%%%
\rule{100mm}{3pt}
\section{\scshape First Semester}
\credit{5}
\lecture{4}
\tutorial{1}
\lab{0}

%\begin{multicols}{\cols}
\add{msc/firstsem}
%\end{multicols}
\newpage
%%%%%%%%%%%%%%%%%%%%%%%%%%%%%%%%%%%%%%%%%%%%%%%%%%%
\rule{100mm}{3pt}
\section{\scshape  Second Semester}
%\begin{multicols}{\cols}
\add{msc/secondsem}
%\end{multicols}
\newpage
%%%%%%%%%%%%%%%%%%%%%%%%%%%%%%%%%%%%%%%%%%%%%%%%%%
\rule{100mm}{3pt}
\section{\scshape  Third Semester}
%\begin{multicols}{\cols}
\add{msc/thirdsem}
%\add{msc/ma507}
%\end{multicols}
%%%%%%%%%%%%%%%%%%%%%%%%%%%%%%%%%%%%%%%%%%%%%%%%%%
\newpage
\rule{100mm}{3pt}
\section{\scshape  Fourth Semester}
%\begin{multicols}{\cols}
\add{msc/fourthsem}
%\end{multicols}
%%%%%%%%%%%%%%%%%%%%%%%%%%%%%%%%%%%%%%%%%%%%

\add{msc/mscopt}

\newpage
\part{\normalsize Ph.D. Applied Mathematics}
\add{phd/structure}
%\begin{multicols}{\cols}
\add{phd/phd}
%\end{multicols}
%



\end{document}
