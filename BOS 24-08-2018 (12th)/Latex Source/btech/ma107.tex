\coursecode{MA 107}
\coursename{Foundation course in Mathematics}
\credit{2}
\lecture{2}
\tutorial{0}
\lab{0}
\section{\courseinfo}
%M. Tech SOBT- 3 years (I semester)
\unit{I}
Representation of real numbers as points on the line and the set of real numbers as complete ordered field, Bounded and unbounded sets, neighborhoods and limit points, Complex Number, D’ Moivre'’s Theorem, Natural Logarithm of complex number, Powers of Complex number Summation of series and complex roots of unity, Binomial Theorem (without infinite series), Algebra of matrices, symmetric and skew symmetric matrices, Hermitian and skew Hermitian matrices, orthogonal matrices, singular and non-singular matrices and their properties. unitary, involutory and nilpotent matrices, Adjoint and inverse of a matrix and related properties. Determinants of Matrices: Definition, properties and applications of determinants for 3rd order  and higher orders, Rank of Matrices, system of linear equations, Cramer'’s rule, eigenvalues and eigenvectors. 

\unit{II}
Limits of functions, Continuity, Differentiation, Successive Differentiation, Expansion of Functions – Rolle'’s theorem, Mean Value theorem, Cauchy mean value theorem, Integration – Definite and Indefinite (ordinary, method of substitution, special trigonometric function, partial fraction) Application of integration to find area, Differential equations --homogeneous and Linear ODE and applications to acceleration and velocity model, growth and decay model. 

\unit{III}
Functions of several variables, Limit and continuity of functions of two or more real variables, Partial Differentiation, Derivative as a slope, higher order derivatives, Leibnitz rule, chain rule, Euler theorem on homogeneous functions and its application.

\begin{thebibliography}{99}
        \bibitem{k} Apostol, Tom. M. (2002): \emph{Calculus, Vol. I.,} John Wiley \& Sons.  
        \bibitem{k} Ross, S.L. (1984): \emph{Differential Equations}, John Wiley and Sons (Student Edition). 
        \bibitem{k} Stewart, J. (2009): \emph{Essential Calculus: Early Transcendentals}, Cengage Publications, 7th Edition
        \bibitem{k} Erwin Kreyszig (2009): \emph{Advanced Engineering Mathematics}, John Wiley and Sons. 
\end{thebibliography}
