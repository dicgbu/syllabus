\vspace{0.1cm}
\coursecode{MA 102}
\coursename{Engineering Mathematics-II}
\credit{4}
\lecture{3}
\tutorial{1}
\lab{0}
\section{\courseinfo}
\unit{I} 
Introduction to matrices, Elementary row and column operations and  reduced echelon forms,  Normal Form, Inverse and Rank of a matrix, Consistency of linear system of equations and their solutions.

\unit{II}
Finite dimensional vector spaces over reals, Subspace, Linear Dependence and Independence of vectors, Basis, Dimension. Linear transformations, range and kernel of a linear map, rank and nullity, Inverse of linear transformation, rank nullity theorem, composition of linear maps, Matrix associated with linear maps. Characteristic equation and characteristic polynomial, eigenvalues and eigenvectors, Cayley-Hamilton theorem, diagonalisation.

\unit{III}
Symmetric and skew symmetric matrices, orthogonal matrices, Inner product spaces, Gram-Schmidt orthogonalization.

\unit{IV}
Functions of complex variables, Limit, Continuity and Differentiability of Complex functions. C-R equation, Analytic function,  Harmonic functions, Elementary (exponential, trigonometric and logarithm). Conformal mapping. Line Integral in complex form, Cauchy's integral theorem, Morera's Theorem,  Cauchy's integral formula: Cauchy's Integral formula for derivatives of analytic functions, Liouville'’s theorem, Maximum-modulus theorem (without proof), Fundamental Theorem of algebra.

\unit{V}
Taylor’s and Laurent'’s Series, Singularities, Zeroes and Poles, Residue, Residue theorem, Evaluation of real integrals.

\paragraph{Textbook}
E. Kreyszig, \emph{Advanced Engineering Mathematics}. John Wiley and Sons, 2005.
\begin{thebibliography}{}
\let\clearpage\relax
\bibitem{3} R. K. Jain and S.R.K Iyengar, \emph{Advanced Engineering Mathematics}, Narosa Publications.
\bibitem{7}     J. W. Brown, R. V. Churchill, \emph{Complex Variables and Application}s, McGraw-Hill Higher Education; 8 edition, 2008. 


\end{thebibliography}














